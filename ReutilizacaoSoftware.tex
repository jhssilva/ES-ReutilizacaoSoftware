\documentclass[runningheads]{llncs}
%
\usepackage[T1]{fontenc}
% T1 fonts will be used to generate the final print and online PDFs,
% so please use T1 fonts in your manuscript whenever possible.
% Other font encondings may result in incorrect characters.
%
\usepackage{graphicx}
% Used for displaying a sample figure. If possible, figure files should
% be included in EPS format.
%
% If you use the hyperref package, please uncomment the following two lines
% to display URLs in blue roman font according to Springer's eBook style:
%\usepackage{color}
%\renewcommand\UrlFont{\color{blue}\rmfamily}
%
\begin{document}
%
\title{Reutilização de Software\thanks{Universidade de Évora.}}
%
%\titlerunning{Abbreviated paper title}
% If the paper title is too long for the running head, you can set
% an abbreviated paper title here
%
\author{Hugo Silva \email{m53080@alunos.uevora.pt}}

%
\authorrunning{H. Silva}
% First names are abbreviated in the running head.
% If there are more than two authors, 'et al.' is used.
%
\institute{Universidade de Évora, Évora, Portugal}
%
\maketitle              % typeset the header of the contribution
%
\begin{abstract}

O presente artigo aborda a reutilização de software como um pilar essencial no desenvolvimento de software. A reutilização de software consiste em reutilizar software existente para criar novos sistemas, permitindo assim, poupar tempo e recursos, aumentando a qualidade de software. É abordado os problemas associados à reutilização de software, como também as formas de minimizar os problemas que podem surgir devido à sua reutilização. O presente artigo aborda também o desenvolvimento de software baseado em componentes e em serviços, como as vantagens e benefícios associados à reutilização de software. Por fim, é realizada a conclusão sobre a reutilização de software.

\keywords{Reutilização de Software  \and Desenvolvimento baseado em componentes e serviços \and Problemas e soluções na reutilização de software.}
\end{abstract}
%
%
%
\section{Introdução}

\section{Reutilização de Software}

A reutilização de Software (Software Engineering (Ian Sommer Ville)) é uma estratégia utilizada em engenharia de software que tem como intuito reutilizar software existente.\par
A estratégia baseada em reutilização de Software têm sido adotada devido à necessidade de:


\begin{itemize}
    \item Menores custos de produção e manutenção de software;
    \item Maior velocidade de entrega dos sistemas informáticos;
    \item Aumento da qualidade do software.
\end{itemize}

As empresas cada vez mais veem o seu software como um ativo importante e têm vindo a  adotar a estratégia de reutilização do software, pois podem aumentar o seu retorno sobre os investimentos no software.\par
As unidades de software que podem ser reutilizadas podem ser de diferentes tamanhos, sendo:

\begin{itemize}
    \item Reutilização completa dos Sistemas;
    \item Reutilização da Aplicação;
    \item Reutilização de Componentes;
    \item Reutilização de Objectos e funções.
\end{itemize}

Todo o software e componentes que incluem funcionalidades genéricas podem ser potencialmente reutilizáveis, no entanto, existem certos softwares e componentes que por vezes são tão específicos que são muito custosos de modificar para a lógica de negócio pretendida.

\subsection{Vantagens e benefícios associados à reutilização de software}

Existem variadíssimas vantagens e benefícios associados à estratégia de reutilização de software, estes são:

\begin{itemize}
    \item Aceleração do desenvolvimento;
    \item Uso efectivo de especialistas;
    \item Maior confiança no software;
    \item Baixo custos de desenvolvimento;
    \item Redução da margem de erro na estimativa dos projectos;
    \item Conformidade com os padrões resulta em maior confiabilidade e menos erros.
\end{itemize}

\subsubsection{Aceleração do desenvolvimento.}

A reutilização de software pode acelerar o desenvolvimento do software, isto porque, o tempo de desenvolvimento e validação podem ser reduzidos. Trazer um sistema para o mercado o mais cedo possível é normalmente mais importante do que os custos totais de desenvolvimento.

\subsubsection{Uso efectivo de especialistas.}

A reutilização de software permite que os especialistas se foquem no desenvolvimento de novo software reutilizável que encapsula os seus conhecimentos, em vez de estarem constantemente a fazer o mesmo trabalho.

\subsubsection{Maior confiança no software.}

A reutilização de software estabelece uma maior confiança no software existente porque o software reutilizável normalmente já foi experimentado e testado anteriormente em sistemas reais, o que permitiu encontrar e corrigir erros de design e de implementação.

\subsubsection{Baixo custos de desenvolvimento.}

A reutilização de software pode permitir baixar os custos de desenvolvimento de software.. Isto porque, é possível reutilizar componentes e serviços já existentes, em vez de desenvolver software do zero. Reduzindo assim o tempo e recursos envolvidos no processo de desenvolvimento.  


\subsubsection{Redução da margem de erro na estimativa dos projectos.}

A reutilização de software ajuda na redução de margem de erro na estimativa dos projectos. Isto porque o software já passou por testes e validação em outros projetos, reduzindo assim o risco de erros ou problemas encontrados.\par
O custo do software existente já é conhecido, enquanto os custos de desenvolvimento estão sempre dependentes sobre o julgamento. Este é um importante factor a ter em consideração na gestão do projecto, porque reduz a margem de erro na estimativa.


\subsubsection{Conformidade com os padrões resulta em maior confiabilidade e menos erros.}

Alguns padrões, como padrões de interfaces, podem ser implementados como componentes reutilizáveis. Por exemplo, se um menu é implementado utilizando um componente reutilizável, todas as aplicações apresentam o mesmo formato de menu ao utilizador. O uso de padrões melhora a confiabilidade e usabilidade, pois, os utilizadores fazem menos erros quando apresentados com uma interface familiar.

\subsection{Problemas associados à reutilização de software}

\subsection{Desenvolvimento de software baseado em componentes}


\subsection{Desenvolvimento de software baseado em serviços}



\subsection{Formas para minimizar os problemas que podem surgir devido à reutilização de software}

\section{Conclusão}


\section{First Section}
\subsection{A Subsection Sample}
Please note that the first paragraph of a section or subsection is
not indented. The first paragraph that follows a table, figure,
equation etc. does not need an indent, either.

Subsequent paragraphs, however, are indented.

\subsubsection{Sample Heading (Third Level)} Only two levels of
headings should be numbered. Lower level headings remain unnumbered;
they are formatted as run-in headings.

\paragraph{Sample Heading (Fourth Level)}
The contribution should contain no more than four levels of
headings. Table~\ref{tab1} gives a summary of all heading levels.

\begin{table}
\caption{Table captions should be placed above the
tables.}\label{tab1}
\begin{tabular}{|l|l|l|}
\hline
Heading level &  Example & Font size and style\\
\hline
Title (centered) &  {\Large\bfseries Lecture Notes} & 14 point, bold\\
1st-level heading &  {\large\bfseries 1 Introduction} & 12 point, bold\\
2nd-level heading & {\bfseries 2.1 Printing Area} & 10 point, bold\\
3rd-level heading & {\bfseries Run-in Heading in Bold.} Text follows & 10 point, bold\\
4th-level heading & {\itshape Lowest Level Heading.} Text follows & 10 point, italic\\
\hline
\end{tabular}
\end{table}


\noindent Displayed equations are centered and set on a separate
line.
\begin{equation}
x + y = z
\end{equation}
Please try to avoid rasterized images for line-art diagrams and
schemas. Whenever possible, use vector graphics instead (see
Fig.~\ref{fig1}).

\begin{figure}
\includegraphics[width=\textwidth]{fig1.eps}
\caption{A figure caption is always placed below the illustration.
Please note that short captions are centered, while long ones are
justified by the macro package automatically.} \label{fig1}
\end{figure}

\begin{theorem}
This is a sample theorem. The run-in heading is set in bold, while
the following text appears in italics. Definitions, lemmas,
propositions, and corollaries are styled the same way.
\end{theorem}
%
% the environments 'definition', 'lemma', 'proposition', 'corollary',
% 'remark', and 'example' are defined in the LLNCS documentclass as well.
%
\begin{proof}
Proofs, examples, and remarks have the initial word in italics,
while the following text appears in normal font.
\end{proof}
For citations of references, we prefer the use of square brackets
and consecutive numbers. Citations using labels or the author/year
convention are also acceptable. The following bibliography provides
a sample reference list with entries for journal
articles~\cite{ref_article1}, an LNCS chapter~\cite{ref_lncs1}, a
book~\cite{ref_book1}, proceedings without editors~\cite{ref_proc1},
and a homepage~\cite{ref_url1}. Multiple citations are grouped
\cite{ref_article1,ref_lncs1,ref_book1},
\cite{ref_article1,ref_book1,ref_proc1,ref_url1}.

%
% ---- Bibliography ----
%
% BibTeX users should specify bibliography style 'splncs04'.
% References will then be sorted and formatted in the correct style.
%
% \bibliographystyle{splncs04}
% \bibliography{mybibliography}
%
\begin{thebibliography}{8}
\bibitem{ref_article1}
Author, F.: Article title. Journal \textbf{2}(5), 99--110 (2016)

\bibitem{ref_lncs1}
Author, F., Author, S.: Title of a proceedings paper. In: Editor,
F., Editor, S. (eds.) CONFERENCE 2016, LNCS, vol. 9999, pp. 1--13.
Springer, Heidelberg (2016). \doi{10.10007/1234567890}

\bibitem{ref_book1}
Author, F., Author, S., Author, T.: Book title. 2nd edn. Publisher,
Location (1999)

\bibitem{ref_proc1}
Author, A.-B.: Contribution title. In: 9th International Proceedings
on Proceedings, pp. 1--2. Publisher, Location (2010)

\bibitem{ref_url1}
LNCS Homepage, \url{http://www.springer.com/lncs}. Last accessed 4
Oct 2017
\end{thebibliography}
\end{document}
